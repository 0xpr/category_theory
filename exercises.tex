\documentclass[11pt]{article}
\usepackage{color}   %May be necessary if you want to colored links
\usepackage{amsmath}
\usepackage{hyperref}
\hypersetup{
  colorlinks=true, %set true if you want colored links
  linktoc=all,     %set to all if you want both sections and subsections linked
  linkcolor=blue,  %choose some color if you want links to stand out
}

\title{\textbf{Category Theory}}
\author{Teja Prabhu}
\date{}

\begin{document}
\maketitle
\newpage
\tableofcontents
\newpage

\section{Categories, Functors, and Natural Transformations}
\newpage

\subsection{Functors}

\underline{Exercise 1} 
\vspace{2mm}

Show how each of the following constructions can be regarded as a functor: 

\vspace{2mm}
\textbf{(a)} \emph{The field of quotients of an integral domain}
\vspace{2mm}

\noindent
\emph{Proof}

Let $IntDom$ be a category with integral domains as objects and \textbf{injective} homomorphisms as arrows; and $Field$ be the category of fields, with arrows obeying ring homomorphism rules. Define:

\begin{itemize}
    \item $T_o$, the object function sends each integral domain to its corresponding field of quotients. 
	\item $T_f$, the arrow function sends each morphism $f$ in $IntDom$ to the induced morphism $T_f(f) : a/b \longmapsto f(a) / f(b)$. 
\end{itemize}

Since, arrows in $IntDom$ are injective, $b \neq 0 \Longrightarrow f(b) \neq 0$. And:

$ a/b \sim a'/b' \Longrightarrow ab' = a'b \Longrightarrow f(a)f(b') = f(a')f(b) \Longrightarrow f(a)/f(b) \sim f(a')/f(b')$. Therefore, $T_f(f)$ is \textbf{well-defined}. The functor rules 
$T_f(f \circ g) = T_f(f) \circ T_f(g)$ and $T_f(id_o) = id_{T_o(o)}$ are satisfied; hence, the construction is a functor. 

\vspace{2mm}

\textbf{(b)} \emph{The Lie Algebra of a Lie Group}

skipped; don't know Differential Geometry

\vspace{2mm}

\noindent
\underline{Exercise 2}
\vspace{2mm}

Show that functors $1 \rightarrow C$, $2 \rightarrow C$, and $3 \rightarrow C$ correspond respectively to objects, arrows, and composable pairs of arrows in $C$.
\vspace{2mm}

Trivial. 
\vspace{2mm}

\noindent
\underline{Exercise 3}
\vspace{2mm}

(\textbf{a}) \emph{A functor between two preorders is a function which is monotonic}

\vspace{2mm}
\noindent
\emph{Proof}

A preorder is a category $P$ in which, given objects $p$ and $p'$, there is at most one arrow $p \rightarrow p'$.

Monotonicity condition: $p \leq q \Longrightarrow T(p) \leq T(q)$. A functor preserves the order of the domain and codomain. Hence a functor between two preorders can be interpreted as a monotonic function. 

\vspace{2mm}
(\textbf{b}) \emph{A functor between two groups is a morphism of groups}

\vspace{2mm}
\noindent
\emph{Proof}

A group is a one object category. So there is only one object function possible. Furthermore, the functor laws correspond exactly to the group homomorphism laws, i.e., $T(f \circ g) = T(f) \circ T(g)$. This works since the arrows of the group category correspond to the elements of the group. We don't even need the functor law regarding identities, since every group has an inverse, i.e., in this case the functor is between two groupoids. 

% exercise 2(c)
\vspace{2mm}
(\textbf{c}) \emph{If G is a group, a functor $G \rightarrow Set$ is a permutation representation
of $G$, while $G \rightarrow Matr_K$ is a matrix representation of $G$}

\vspace{2mm}
\noindent
\emph{Proof}

Since $G$ is a one object category, the object function will map it to some set $S$ in $Set$. The arrow function will: 
\begin{itemize}
	\item send the identity $e$ of the group to the identity function on $S$
	\item since every element in a group has an inverse, every function on $S$ should have an inverse too. If $a \cdot b = e$, in $G$, then since $T(a \circ b) = T(e) = id_S = T(a) \circ T(b)$. So every arrow $T(a)$ will have an inverse. So the elements of the group map to bijections from $S$ to itself.  
\end{itemize}

From the previous two points, we have by definition that a functor from $G$ to $Set$ is a permutation representation of $G$.

\vspace{2mm}

A functor from $G$ to $Matr_K$:

\begin{itemize}
	\item The object function will map the single object to some positive integer $n$.
	\item The arrow function will map elements of the group to some $n \times n$ matrix, such that functor laws are satisfied. This means that it will only map to non-singular matrices, which obey the group homomorphism laws with respect to matrix multiplication.
\end{itemize}

Since, matrices are nothing but linear transformations over vector spaces, functors from $G$ to $Matr_K$ will correspond to matrix representations of $G$. 

\vspace{2mm}

\noindent
\underline{Exercise 4}
\vspace{2mm}

Prove that there is no functor $Grp \rightarrow Ab$ sending each group $G$ to its center. 

\vspace{2mm}
\noindent
\emph{Proof}

Consider the symmetric groups $S_2$ and $S_3$. Let: 
\begin{itemize}
	\item $f : S_2 \rightarrow S_3$ be the inclusion homomorphism. Since in $S_3$ if we keep $3$ fixed, we'll find that subgroup of $S_3$ to be isomorphic to $S_2$. 
	\item $g: S_3 \rightarrow S_2$, such that the kernel of $g$ is $A_3$.  
	\item The center of $S_2$ is itself.
	\item The center of $S_3$ is the trivial subgroup. 
\end{itemize}

Then $g \circ f = id_{S_2}$. Let $T_o$ be any object function which sends each group to its center. Then, $T_f(f)$ must be the zero map, as the center of $S_3$ is the trivial subgroup. Therefore, for all maps $g$ from $Z(S_3)$ to $Z(S_2)$, $T_f(g)$ must be the zero map. Hence, $T_f(g \circ f)$ will be the zero homomorphism. But functors should take identities to identities. So, we cannot construct any arrow function which will obey the functor rules. 

\vspace{2mm}
\noindent
\underline{Exercise 5}
\vspace{2mm}

Find two different functors $T: Grp \rightarrow Grp$ with object function $T(G) = G$ the identity for every group $G$.

\vspace{2mm}
\noindent
\emph{Proof}

One is the identity functor. For the other functor pick a group $G$ and an automorphism $\phi$ on $G$. 
$G' \xrightarrow{f} G \xrightarrow{g} G''$. Replace all such $f$'s with $\phi \circ f$ and all such $g$'s with $g \circ \phi^{-1}$, and leave all other arrows in the category unchanged. This will form another functor. 






\newpage
\subsection{Natural Transformations}

% ------------------ %
\vspace{2mm}
\noindent
\underline{Exercise 1}
\vspace{2mm}

Let $S$ be a fixed set, and $X^S$ the set of all functions $h: S \rightarrow X$. Show that $X \mapsto X^S$
is the object function of a functor $Set \rightarrow Set$, and that evaluation $e_x : X^S \times S \xrightarrow{.} X$,
defined by $e(h, s) = h(s)$, the value of the function $h$ at $s \in S$, is a natural transformation.

\vspace{2mm}
\noindent
\emph{Proof}

$e_x$ forms a natural transformation among the identity functor and the functor constructed below:
\begin{itemize}
	\item object function: $X \mapsto X^S \times S$
	\item arrow function: $f \mapsto (f \circ g) \times id$ for all $g \in X^S$
\end{itemize}

\vspace{2mm}
\noindent
\underline{Exercise 2}
\vspace{2mm}

If $H$ is a fixed group, show that $G \mapsto H \times G$ defines a functor $H \times - : Grp \to Grp$,
and that each morphism $f : H \to K$ of groups defines a natural transformation $H \times - \xrightarrow{.} K \times -$.

\vspace{2mm}
\noindent
\emph{Proof}

Let $f : H \to K$ be a morphism. Let $\phi : G \to G'$ be a morphism between two groups. Then, $(h, g) \mapsto (h, \phi(g))$ in the $H \times$ --- functor; and $(k, g) \mapsto (k, \phi(g))$ in the $K \times$ --- functor. 

Now if $f$ is any morphism from $H$ to $K$, then 

$(h, g) \mapsto (h, \phi(g)) \mapsto (f(h), \phi(g))$ and $(h, g) \mapsto (f(h), g) \mapsto (f(h), \phi(g))$ map to the same element, which implies that the diagram is commutative. Hence, it is a natural transformation.



\vspace{2mm}
\noindent
\underline{Exercise 3}
\vspace{2mm}

If $B$ and $C$ are groups (regarded as categories with one object each) and
$S, T: B \to C$ are functors (homomorphisms or groups), show that there is a
natural transformation $S \xrightarrow{.} T$ if and only if $S$ and $T$ are conjugate; i.e., if and
only if there is an element $h \in C$ with $Tg = h(Sg)h^{-1}$ for all $g \in B$.

\vspace{2mm}
\noindent
\emph{Proof}

Let's assume there is a natural transformation from $S \xrightarrow{.} T$. Since, there is only one object in each of the categories, there will only be one arrow in the natural transformation that we have to think about -- let's say $h \in C$. 

From the natural transformation commutative condition we have: $h \circ Sg = Tg \circ h$ for all arrows $g \in B$.

$ \implies Tg = h(Sg)h^{-1}$ for each $g \in B$, we can do this since the category is actually a groupoid.

$ \implies h$ and $g$ are conjugate.

$S$ and $T$ are conjuagte $\implies \exists h . Tg = h (Sg) h^{-1}$

Since, $h \in C$, $h$ is an arrow of $C$, so we can pick this to be the arrow of the natural transformation (we can also pick $h^{-1}$).



\vspace{2mm}
\noindent
\underline{Exercise 4}
\vspace{2mm}

For functors $S, T: C \to P$ where $C$ is a category and $P$ a preorder, show that
there is a natural transformation $S \xrightarrow{.} T$ (which is then unique) if and only if
$Sc \leq Tc$ for every object $c \in C$.

\vspace{2mm}
\noindent
\emph{Proof}

For all $c \in C$, there is the identity arrow. So, if we draw the commutative diagram with $c' = c$, then for every object we have $Sc \leq Tc$ if there is a natural transformation from $S$ to $T$.

Conversely, if $Sc \leq Tc$ for every $c$, then there will be exactly one arrow from $Sc \to Tc$ for all $c$. Since in a preorder between any two objects there can be at most one arrow, the diagram will commute, and the arrow defined above will be the natural transformation.


\vspace{2mm}
\noindent
\underline{Exercise 5}
\vspace{2mm}

Show that every natural transformation $\tau : S \xrightarrow{.} T$ defines a function (also called $\tau$)
which sends each arrow $f: c \to c'$ of $C$ to an arrow $\tau f: Sc \to Tc'$ of $B$ in such a
way that $Tg \circ \tau f = \tau(gf) = \tau g \circ Sf$ for each composable pair $<g, f>$. Conversely,
show that every such function $\tau$ comes from a unique natural transformation
with $\tau_c = \tau(1_c)$ (This gives an "arrows only" description of a natural transformation.)

\vspace{2mm}
\noindent
\emph{Proof}

Let $\tau(f) = T(f) \circ \tau_c = \tau_{c'} \circ S(f)$, then the relations are satisfied because the diagram is commutative. 

Conversely, let there be such a $\tau$. Then, consider $c \xrightarrow{1_c} c \xrightarrow{f} c' \xrightarrow{1_{c'}} c'$. 

$\tau(f \circ 1_c) = \tau(1_{c'} \circ f)$

$\implies T(f) \circ \tau_c = \tau_{c'} \circ S(f)$, which is the definition of a natural transformation. The transformation is unique because $\tau$ is a function and $\tau(1_c)$ and $\tau(1_c')$ have only one value each.


\vspace{2mm}
\noindent
\underline{Exercise 6}
\vspace{2mm}

Let $F$ be a field. Show that the category of all finite-dimensional vector spaces
over $F$ (with morphisms all linear transformations) is equivalent to the category
$Matr_F$ described in §2.

\vspace{2mm}
\noindent
\emph{Proof}

\begin{itemize}
	\item object function: $V^n \mapsto n$
	\item arrow function: takes $L : V^n \to V^m \mapsto A_{m \times n}$, where the matrix is the transformation matrix.  
\end{itemize}

We can pick the natural transformation to be $V^n \mapsto n$. Since this is a bijection, there is a natural equivalence between the identity functor on the category of vector spaces and $Matr_F$. Hence, both these categories are equivalent.


\newpage
\subsection{Monics, Epis, and Zeros}


\vspace{2mm}
\noindent
\underline{Exercise 1}
\vspace{2mm}

Find a category with an arrow which is both epi and monic, but not invertible.

\vspace{2mm}
\noindent
\emph{Proof}

3Let $R$ be a one object category with arrows all differentiable functions from $R$ to itself. Then, consider $x \mapsto x^3$, which is a bijective function, so it is both monic and epi, but is not invertible since $x \mapsto x^{1/3}$ is not differentiable at $0$.



\vspace{2mm}
\noindent
\underline{Exercise 2}
\vspace{2mm}

Prove that the composite of monics is monic, and likewise for epis.

\vspace{2mm}
\noindent
\emph{Proof}

$g \circ f$ is monic if $g, f$ are monic. Let $p$ and $q$ be two functions whose codomain is the same as the domain of $g \circ f$. Then, 

$(g \circ f) \circ p = (g \circ f) \circ q$

$\implies g \circ (f \circ p) = g \circ (f \circ q)$

$\implies f \circ p = g \circ q$ (since $g$ is monic)

$\implies p = q$ (since $f$ is monic)

$\implies g \circ f$ is monic.


$g \circ f$ is epi if $g, f$ are epi. Let $p$ and $q$ be 2 functions whose domain is the same as the codomain of $g \circ f$. Then,

$p \circ (g \circ f) = q \circ (g \circ f)$

$\implies (p \circ g) \circ f = (q \circ g) \circ f$

$\implies p \circ g = q \circ g$ (since $f$ is epi)

$\implies p = q$ (since $g$ is epi)

$\implies g \circ f$ is epi.

\vspace{2mm}
\noindent
\underline{Exercise 3}
\vspace{2mm}

If a composite $g \circ f$ is monic, so is $f$. Is this true of $g$?

\vspace{2mm}
\noindent
\emph{Proof}

Let's say $f$ is not monic. Then there are functions $x$ and $x'$ not equal, such that:

$f \circ x = f \circ x'$

$\implies (g \circ f) \circ x = (g \circ f) \circ x'$

$\implies x = x'$ ($g \circ f$ is monic)

which is a contradiction. So $f$ is monic. It's not necessarily true for $g$. For example, in $Set$, $g$ has to be injective only in the image of $f$, and outside this set it may not be injective.


\vspace{2mm}
\noindent
\underline{Exercise 4}
\vspace{2mm}

Show that the inclusion $Z \to Q$ is epi in the category $Rng$.

\vspace{2mm}
\noindent
\emph{Proof}

Let $Z \xrightarrow{i} Q \xrightarrow{\phi, \psi} R$

In such a case, we must prove that if $\phi \circ i = \psi \circ i \implies \phi = \psi$. Let $\frac{p}{q} \in Q $, then:

$\phi(\frac{p}{q}) \neq \psi(\frac{p}{q})$ 

$\implies \phi(\frac{1}{q}) \neq \psi(\frac{1}{q})$ (since $\phi$ and $\psi$ will be the same on $\mathbf{Z}$)

$\implies \phi(1) \neq \psi(1)$ (multiplying both sides by $\phi(q) = \psi(q)$), which is a contradiction.

Hence, $\phi = \psi$, and the inclusion is epi in the category $Rng$.

\vspace{2mm}
\noindent
\underline{Exercise 5}
\vspace{2mm}

In $Grp$ prove that every epi is surjective.

\vspace{2mm}
\noindent
\emph{Proof}

Let $\phi : G \to H$ be an epi in $Grp$. Consider the order $k$ of $H / Im(\phi)$:

\begin{itemize}
	\item If $k = 2$, then we know that $Im(\phi)$ is a normal subgroup. Let $\pi : H \to H / Im(\phi)$ be the canonical map, i.e., $h \mapsto h + Im(\phi)$; and let $z : H \to H / Im(\phi)$ be the zero map. Then, $\pi \circ \phi = z \circ \phi$. But, since $\phi$ is epi, $\pi = z \implies z$ is surjective $\implies Im(\phi) = H \implies k = 1$, which is a contradiction.
	\item If $k > 2$, then we can pick three unique cosets $Im(\phi), u + Im(\phi), v + Im(\phi)$. Let $Perm H$ be the group of all permutations of the set $H$. Define a map $\lambda \in Perm H$, such that $\lambda (xu) = xv$ and $\lambda (xv) = xu$ for all $x \in Im(f)$ and equal to the identity function everywhere else. Let $\psi: H \to Perm H$ send each $h$ to left multiplication $\psi_h$ by $h$, and let $\psi_{h}' = \lambda^{-1} \psi_h \lambda$. Then, $\psi \phi = \psi' \phi$, but $\psi \neq \psi'$, contradicting $\phi$ is epi.
\end{itemize}

Hence, from the above two points $k = 1$ is the only valid value, which implies that $\phi$ is surjective.


\vspace{2mm}
\noindent
\underline{Exercise 6}
\vspace{2mm}

In $Set$, show that all idempotents split.

\vspace{2mm}
\noindent
\emph{Proof}

Let $f : b \to b$ be idempotent. $f^2 = f \implies f$ is the identity on $Im(f)$. Define $h$ be the identity on $Im(f)$ and $g$ be $f$. Then $f = hg$ and $gh = 1$. Hence, all idempotents split in $Set$.
 we

\vspace{2mm}
\noindent
\underline{Exercise 7}
\vspace{2mm}

An arrow $f: a \to b$ in a category $C$ is \emph{regular} when there exists an arrow $g: b \to a$
such that $f g f = f$. Show that $f$ is regular if it has either a left or a right inverse,
and prove that every arrow in $Set$ with $a \neq \emptyset$ is regular.

\vspace{2mm}
\noindent
\emph{Proof}

\begin{itemize}
	\item If it has a left inverse, then there is an $r$ such that $r \circ f = 1_a$. Choosing $g = r$, we have $f g f = f$
	\item If it has a right inverse, then there is an $r$ such that $f \circ r = 1_b$. Choosing $g = r$, we again have $f g f = f$
\end{itemize}

(will prove 2nd part later)


% ------------------ %
\end{document}






































