\noindent
\underline{Exercise 3}
\vspace{2mm}

If $B$ and $C$ are groups (regarded as categories with one object each) and
$S, T: B \to C$ are functors (homomorphisms or groups), show that there is a
natural transformation $S \xrightarrow{.} T$ if and only if $S$ and $T$ are conjugate; i.e., if and
only if there is an element $h \in C$ with $Tg = h(Sg)h^{-1}$ for all $g \in B$.

\vspace{2mm}

\noindent
\emph{Proof}

Let's assume there is a natural transformation from $S \xrightarrow{.} T$. Since, there is only one object in each of the categories, there will only be one arrow in the natural transformation that we have to think about -- let's say $h \in C$. 

From the natural transformation commutative condition we have: $h \circ Sg = Tg \circ h$ for all arrows $g \in B$.

$ \implies Tg = h(Sg)h^{-1}$ for each $g \in B$, we can do this since the category is actually a groupoid.

$ \implies h$ and $g$ are conjugate.

$S$ and $T$ are conjuagte $\implies \exists h . Tg = h (Sg) h^{-1}$

Since, $h \in C$, $h$ is an arrow of $C$, so we can pick this to be the arrow of the natural transformation (we can also pick $h^{-1}$).

\vspace{2mm}
