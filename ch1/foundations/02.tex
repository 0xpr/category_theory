\noindent
\underline{Exercise 2}
\vspace{2mm}

\textbf{(a)} Given a universe $U$ and a function $f: I \to b$ with domain $I \in
U$, show that the usual union $\bigcup_i f_i$ is a set of $U$.

\vspace{2mm}

\noindent
\emph{Proof}

We must assume $b \subset U$ to prove this. We can employ the same reasoning we
used in the previous proof.

\vspace{2mm}

\textbf{(b)} Show that this one closure property of $U$ may replace condition
\textbf{(v)} and the condition $x \in U$ implies $\cup x \in U$ in the
definition of a universe.

\vspace{2mm}

\noindent
\emph{Proof}

\vspace{2mm}
\emph{I} \textbf{(a)} $implies$ \textbf{(v)} 

$f : I \to b$, $b \subset U$ and $f$ is surjective. Then, from \textbf{(a)}, we
know that $\cup b \in U$. 

$\implies \mathcal{P}(\cup b) \in U$

$\implies \mathcal{P}(\mathcal{P}(\cup b)) \in U$

But, since $b \in \mathcal{P}(\mathcal{P}(\cup b)) \in U$, we have $b \in U$,
from \textbf{(i)}.


\vspace{2mm}
\emph{II} \textbf{(a)} $implies$ \textbf{(iii) b}

Let $b \in U$. Consider the identity function $i: b \to U$. 
Then $x \in b \in U \implies x \in U$, from \textbf{(i)}.
So it is a valid function. Now $Im(i) = b$. From \textbf{(a)}, we know that
$\bigcup Im(i) \in U$, and so, $\cup b \in U$.

\vspace{2mm}
